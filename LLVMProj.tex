\newpage

\section{The LLVM project}

LLVM is an open-source framework providing a modern collection of modular and reusable
compiler and toolchain technologies [20]. The project stemmed from the work of Chris
Lattner, who first implemented core elements of LLVM to support the research of his
master thesis in 2002 [26]. One of the key strengths of LLVM is that it faces active
development from an expert community of contributors, and is widely used across the
industry and academia alike [11]. The project received the ACM Software System Award
in 2012 [4] as an acknowledgement of its contribution to compiler research and implementation.

LLVM officially acknowledges more than 10 main sub-projects [20], which range in diversity from a debugger [9] to a symbolic execution tool [8]. In addition to these, the official
website presents a long list of miscellaneous projects that are based on components of
the LLVM infrastructure [19].

All projects which are part of the LLVM ecosystem are built upon the core libraries, which
are arguably the centrepiece of LLVM. They host the source- and target-independent
optimizer (Section 3.3), the implementation of the LLVM intermediate representation
(Section 3.2), and a suite of command-line tools useful for code manipulation (Section 3.4).
The core libraries also implement various back-end passes that translate IR to machine
code for different platforms (x86, PowerPC, Nvidia GPUs).
For building a complete compiler for a source language, the LLVM core libraries readily
provide the optimizing pass and code generation for common architectures, the frontend being the only missing component. Clang is by far the most prominent front-end
implemented in LLVM [5], and it targets the family of C languages (C/C++ and Objective-C/C++). Coupled with the core libraries, Clang is a powerful compiler
producing high-performance code, and positions itself as a direct competitor to both gcc
and the Intel compiler.

