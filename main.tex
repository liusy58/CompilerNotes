\documentclass[
%draft%     uncomment to activate draft mode (see preamble/proofs)
]{article}   

% preamble -- do not rearrange order of \includes
%\include{classoptions}
%\include{pagesize}
%\include{packages}
%\include{encoding}         
%\include{fonts}
%\include{ToC}
%\include{contributor}
%\include{copyright}
%\include{bibtex}
%\include{environments}
%\include{sectionoptions}
%\include{headerfooter}
%\include{footnoteformat}
%\include{codesnipets}
%\include{proofs}
\usepackage{amsmath} % for align
\usepackage{subcaption}
\captionsetup{compatibility=false}
\usepackage{algorithm}% http://ctan.org/pkg/algorithms
\usepackage{algpseudocode}% http://ctan.org/pkg/algorithmicx
\usepackage[style=numeric,sorting=none]{biblatex}
\addbibresource{main.bib} %Import the bibliography file
\usepackage{tikz}
\usepackage{graphicx}
\usepackage[export]{adjustbox}
\usepackage{caption}
\usepackage{amssymb}
\usepackage{float}
% \usepackage{subfig}
\usepackage{placeins}
\usepackage{listings}
\lstset{
  basicstyle=\ttfamily,
  mathescape
}
%\usepackage{minted}
\usetikzlibrary{shapes}
\usetikzlibrary {positioning}
\usetikzlibrary{chains}
\usetikzlibrary{fit}
\usetikzlibrary{chains,shadows.blur}
\usepackage{geometry}
\usepackage{array}
\usepackage{hyperref}
\usepackage{indentfirst}
\usepackage{pdfpages}


\hypersetup{
    colorlinks=true,
    linkcolor=magenta,
    filecolor=cyan,      
    urlcolor=blue,
}
\graphicspath{ {./images/} }


\usepackage{listings}
\usepackage{xcolor}

\usepackage[autosize]{dot2texi}
\usepackage{tikz}
\usetikzlibrary{shapes,arrows}

\definecolor{codegreen}{rgb}{0,0.6,0}
\definecolor{codegray}{rgb}{0.5,0.5,0.5}
\definecolor{codepurple}{rgb}{0.58,0,0.82}
\definecolor{backcolour}{rgb}{0.95,0.95,0.92}
\usepackage{tcolorbox}

\newtcolorbox{note}[1]{colback=red!5!white,colframe=red!75!black,fonttitle=\bfseries,title=#1}


\newtcolorbox{definition}[1]{colback=blue!5!white,colframe=blue!75!black,fonttitle=\bfseries,title=#1}

\lstdefinestyle{mystyle}{
    % backgroundcolor=\color{backcolour},   
    commentstyle=\color{codegreen},
    keywordstyle=\color{magenta},
    numberstyle=\tiny\color{codegray},
    stringstyle=\color{codepurple},
    basicstyle=\ttfamily\footnotesize,
    breakatwhitespace=false,         
    breaklines=true,                 
    captionpos=b,                    
    keepspaces=true,                 
    numbers=left,                    
    numbersep=5pt,                  
    showspaces=false,                
    showstringspaces=false,
    showtabs=false,                  
    tabsize=2
}

\newtcolorbox{proof}[1]{colback=white,colframe=gray,fonttitle=\bfseries,title=#1}


% \lstset{style=mystyle}


  \lstset{ %
    language=Octave,                % the language of the code
    basicstyle=\footnotesize,           % the size of the fonts that are used for the code
    numbers=left,                   % where to put the line-numbers
    numberstyle=\tiny\color{gray},  % the style that is used for the line-numbers
    stepnumber=2,                   % the step between two line-numbers. If it's 1, each line 
                                    % will be numbered
    numbersep=5pt,                  % how far the line-numbers are from the code
    backgroundcolor=\color{white},      % choose the background color. You must add \usepackage{color}
    showspaces=false,               % show spaces adding particular underscores
    showstringspaces=false,         % underline spaces within strings
    showtabs=false,                 % show tabs within strings adding particular underscores
    frame=single,                   % adds a frame around the code
    rulecolor=\color{black},        % if not set, the frame-color may be changed on line-breaks within not-black text (e.g. commens (green here))
    tabsize=2,                      % sets default tabsize to 2 spaces
    captionpos=b,                   % sets the caption-position to bottom
    breaklines=true,                % sets automatic line breaking
    breakatwhitespace=false,        % sets if automatic breaks should only happen at whitespace
    title=\lstname,                 % show the filename of files included with \lstinputlisting;
                                    % also try caption instead of title
    keywordstyle=\color{blue},          % keyword style
    commentstyle=\color{dkgreen},       % comment style
    stringstyle=\color{mauve},         % string literal style
    escapeinside={\%*}{*)},            % if you want to add LaTeX within your code
    morekeywords={*,...}               % if you want to add more keywords to the set
}


% define issue details
\title{Compiler Optimization Notes}
\newcommand\thejournalsubtitle{Notes for the Compiler Optimization Techniques}
\newcommand\thevolume{}
\newcommand\theseason{May}
\newcommand\theyear{2022}
\newcommand\theissue{\thejournal \ \thevolume \ (\theyear)} 

\newcommand\generaleditor{}
\newcommand\associateeditor{}
\sloppy
\newcommand\thewebsite{https://github.com/liusy58/CompilerNotes}

\begin{document}
\sloppy                         % preferences more space between words over overrunning margins
\lefthyphenmin=3                % suppresses hyphenation after only 1 or 2 characters
                                % NB: You will need to repeat \lefthyphenmin in the text if you use \selectlanguage
%\include{editorialboard}
%\include{titlepage}
%\include{colofon}
\pagenumbering{roman}           
%\tableofcontents  
\thispagestyle{empty}

\maketitle
\tableofcontents

% \include{essays/preface}
\pagenumbering{arabic}


% 
\section{Foundations of Data Flow Analysis}


$\leq$ means more conservative, but not means subset.


\subsection{Transfer Functions}





\subsubsection{monotonicity}

% \begin{tcolorbox}
% efjer
% \end{tcolorbox}

\begin{note}{Monotone framework doesn't mean $ f(x) \leq x$}
For example, reaching definition for just one definition \texttt{a=1} in a BasicBlock(\texttt{BB1}).  \[IN(BB1) = \{\} = x = \top , OUT(BB1) = {a} = f(x) \]
However, $x = \top \leq  f(x)$
\end{note}

\subsubsection{Distributivity}

Not a requirement.

\begin{definition}{Distributivity}

A framework $F,V, \wedge$ is \emph{distributive} is and only if 

\[f(x \wedge y)=f(x) \wedge f(y)\]

which means applying $f$ to the merge input is equal to applying $f$ individually then merge result. 
\end{definition}

Reaching definition is distributive.


Constant Propagation is not distributive. 
\begin{figure}[h]
    \centering
    \includegraphics[width=0.2\textwidth]{CDp.png}
    \caption{}
    \label{fig:p15}
\end{figure}

\subsection{Data Flow Analysis}

\begin{definition}{Definition}
Let $f_1, \dots , f_m \in F$, where $f_i$ is the transfer function for node $i$. $f_p=f_{n_k} \cdot \ldots \cdot f_{n_1}$, where $p$ is a path through nodes $n_1  \cdot \ldots \cdot n_k$. $f_p =$ identify function, if $p$ is an empty path.
\end{definition}

\subsubsection{Precision}

Ideally for each node n, the IN should be $ \wedge f_{p_i}(\top)$ for all possibly executed path $p_i$ reaching n. But determining all possible executed paths is undecidable. Look at the example shown in \ref{fig:p20}.


\begin{figure}[h]
    \centering
    \includegraphics[width=0.2\textwidth]{p20.png}
    \caption{}
    \label{fig:p20}
\end{figure}

So in reality,  we will conservatively include some paths that will never be executed. From a correctness standpoint, this is fine because we will just get an more conservative answer.

\subsubsection{Meet-Over-Path(MOP)}

\begin{definition}{MOP}
For each node n, MOP(n) = $ \wedge f_{p_i}(\top)$ for all possibly executed path $p_i$ reaching n. 

Strictly speaking, MOP considers more paths than necessary, which means 

\[ \textit{MOP = Perfect-Solution} \wedge  \textit{Solution-to-Unexecuted-Paths.}\]

So 

\[ MOP \leq \textit{ Perfect-Solution} \]

MOP is more conservative. 

\end{definition}



\subsection{Solving Data Flow Equations}

Any solution that satisfies equations is a Fixed Point Solution(FP).


\subsubsection{Iterative algorithm }

If framework is monotone and algorithm coverges, then it computes Maximum Fixed Point(MFP).


FP $\leq$ MFP $\leq$ MOP $\leq$ Perfect-solution


Reaching Definition example:
\begin{figure}[h]
    \centering
    \includegraphics[width=0.2\textwidth]{p21.png}
    \caption{}
    \label{fig:p21}
\end{figure}




\subsection{Precision}

If data flow framework is distributive, then if the algorithm converges, $IN[b] = MOP[b]$

A Monotone but not distributive example: Constant Propagation.(Behaves as if there are additional paths)



\subsection{Convergence}
Properties are needed to guarantee convergence:

\begin{itemize}
    \item monotone
    \item finite descending chain
\end{itemize}



\subsection{Speed of Convergence}

\subsubsection{Reverse Post order}

\begin{figure}[h]
    \centering
    \includegraphics[width=0.2\textwidth]{p22.png}
    \caption{}
    \label{fig:p22}
\end{figure}


\subsubsection{Depth-First Iterative Algorithm(forward)
}



\begin{figure}[h]
    \centering
    \includegraphics[width=0.2\textwidth]{p23.png}
    \caption{}
    \label{fig:p23}
\end{figure}


\subsubsection{Cost}

Number of iterations = number of back edges in any acyclic path +2 





% \section{More Examples of Data Flow Analysis: Global Common Sub-expression Elimination; Constant Propagation/Folding}

If we care about the past, what happened before, then it is a forward problem (entry). 


\subsection{Available Expression Analysis}


\begin{definition}{Availability of an Expression E at point P}
E is available at P if every path to P in the flow graph
\begin{itemize}
    \item E must be calculated at least once
    \item no variable in E redefined after the last evaluation

\end{itemize}
\end{definition}


\begin{figure}[h]
    \centering
    \includegraphics[width=0.3\textwidth]{p24.png}
    \caption{}
    \label{fig:p24}
\end{figure}


\subsubsection{Examples}



In \ref{fig:p24} $a-b$, $c+d$ is not available at the last BB. But $x+y$ is.


\begin{figure}[h]
    \centering
    \includegraphics[width=0.3\textwidth]{p25.png}
    \caption{}
    \label{fig:p25}
\end{figure}


In \ref{fig:p25} , $4*i$ is available for both cases.



\begin{figure}[h]
    \centering
    \includegraphics[width=0.3\textwidth]{p26.png}
    \caption{}
    \label{fig:p26}
\end{figure}


In \ref{fig:p26}, we show that calculate transfer functions for complete basic blocks by composing individual instruction transfer functions.


\begin{figure}[h]
    \centering
    \includegraphics[width=0.3\textwidth]{p27.png}
    \caption{}
    \label{fig:p27}
\end{figure}


\subsection{Eliminating CSEs}


\begin{itemize}
    \item Step1: Value Numbering
    \item Step2: Available expression
    \item Step3: If CSE is an "available expression", then transform the code.
    
\end{itemize}

\begin{figure}[h]
    \centering
    \includegraphics[width=0.3\textwidth]{p28.png}
    \caption{}
    \label{fig:p28}
\end{figure}

If we only use value numbering to eliminate common expression in \ref{fig:p28}, we will see that this will just add a lot of new work and no income. But if we calculate Available expression in \ref{fig:p29}, we can find that $x+y$ is such one and can do some optimization.


\begin{figure}[h]
    \centering
    \includegraphics[width=0.3\textwidth]{p29.png}
    \caption{}
    \label{fig:p29}
\end{figure}

\begin{note}{How to deal with Textually identical expression?\ref{fig:p30}}
Just sort the operands.



But for textually different expressions that may be equivalent \ref{fig:p31}, we had better do copy propagation first.

\end{note}
\begin{figure}[h]
    \centering
    \includegraphics[width=0.3\textwidth]{p30.png}
    \caption{}
    \label{fig:p30}
\end{figure}

\begin{figure}[h]
    \centering
    \includegraphics[width=0.3\textwidth]{p31.png}
    \caption{}
    \label{fig:p31}
\end{figure}


\subsubsection{Summary}

\begin{figure}[h]
    \centering
    \includegraphics[width=0.3\textwidth]{p32.png}
    \caption{}
    \label{fig:p32}
\end{figure}


\subsection{Constant Propagation/Folding}

\begin{figure}[h]
    \centering
    \includegraphics[width=0.3\textwidth]{p33.png}
    \caption{}
    \label{fig:p33}
\end{figure}

\subsubsection{Meet Operator in Table Form}
\begin{figure}[h]
    \centering
    \includegraphics[width=0.3\textwidth]{p34.png}
    \caption{}
    \label{fig:p34}
\end{figure}


\subsubsection{Example}

\begin{figure}[h]
    \centering
    \includegraphics[width=0.3\textwidth]{p35.png}
    \caption{}
    \label{fig:p35}
\end{figure}

On the other path in \ref{fig:p35}, x is uninitialized. When we have undefined behavior, hopefully the front end of the compiler should complain about it, but if it doesn't, the optimizer is free to do whatever it wants to do.


\subsubsection{Transfer Function}


\begin{figure}[h]
    \centering
    \includegraphics[width=0.3\textwidth]{p36.png}
    \caption{}
    \label{fig:p36}
\end{figure}



It is not distributive in \ref{fig:p37}.

\begin{figure}[h]
    \centering
    \includegraphics[width=0.3\textwidth]{p37.png}
    \caption{}
    \label{fig:p37}
\end{figure}



\subsection{Copy Propagation}

\subsection{Dead Code Elimination}




\section{Local Optimizations}


\subsection{Basic Blocks}






\section{Introduction to Data Flow Analysis}

\subsection{Structure of data flow analysis}

\subsubsection{What is Data Flow Analysis?}

Local Optimizations only consider optimizations within a node in CFG. 
Data flow analysis will take edges into account, which means composing 
effects of basic blocks to derive information at basic block boundaries.



Typically, we will do local optimization for the first step to know what happens in a 
basic block, step 2 is to do data flow analysis. In he third step, we will go back and 
revisit the individual instructions inside of the blocks.


Data flow analysis is \textbf{flow-sensitive}, which means we take into account
 the effect of control flow. It is also a \textbf{intraprocedural analysis} which means
 the analysis is within a procedure. Data-flow analysis computes its solutions over the paths in
 a control-flow graph. The well-known, meet-over-all-paths
 formulation produces safe, precise solutions for general dataflow problems. All paths-whether feasible or infeasible,
 heavily or rarely executed-contribute equally to a solution. 

Here are some examples of intraprocedural optimizations:

\begin{itemize}
\item \textbf{constant propagation}. Constant propagation is a well-known global flow analysis 
problem. The goal of constant propagation is to discover values that are constant on all possible 
executions of a program and to propagate these constant values as far forward through the program 
as possible. Expressions whose operands are all constants can be evaluated at compile time and the 
results propagated further.

\item \textbf{common subexpression elimination}

\item \textbf{dead code elimination}. Actually, source code written by programmers doesn't contain
 a lot of dead code, dead code happens to occur partly because of how the front end translates code into 
 the IR. Doing optimizations will also turn code into dead.

\end{itemize}

% \subsection{Static    Program    vs.    Dynamic    Execution }

% Static program 




\subsubsection{Static Program vs. Dynamic Execution}


Program is statically infinite, but there can be infinite many dynamic execution paths. On one hand, analysis
 need to be precise, so we will take into account as much dynamic execution as possible. On the other hand, analysis
 need to do the analysis quickly. For a compromise, the analysis result is \textbf{conservative} and what it does id for each 
 point in the program, combines information of all the instances of the same program point.


\subsubsection{Data Flow Analysis Schema}


\subsection{Reaching Definitions}

The Reaching Definitions Problem is a data-flow problem used to answer the
following questions: Which definitions of a variable \textit{X} reach a given use of \textit{X} in
an expression? Is \textit{X} used anywhere before it is defined? A definition\textit{d} reaches a point \textit{p} if there exists path 
from the point immediately following \textit{d} to \textit{p} such that \textit{d} is not killed(overwritten) along that path.





\subsubsection{Example}





\subsection{}


\section{ Live Variabl Analysis   }

In compilers, live variable analysis (or simply liveness analysis)
 is a classic data-flow analysis to calculate the variables that 
 are live at each point in the program. A variable is live at 
 some point if it holds a value that may be needed in the future, 
 or equivalently if its value may be read before the next time 
 the variable is written to. \footnote{based on Wikipedia}

\subsection{Motivation}

Programs may contain 

\begin{itemize}
\item code which gets executed but which has no useful
effect on the program's overall result;
\item occurrences of variables being used before they
are defined;
\item many variables which need to be allocated
registers and/or memory locations for compilation.

\end{itemize}

The concept of variable liveness is useful in dealing 
with all three of these situations.



\subsection{Problem formulation}
Liveness is a data-flow property of variables:
“Is the value of this variable needed?” We therefore 
usually consider liveness from an instruction's 
perspective: each instruction (or node of the
flowgraph) has an associated set of live variables.


\subsection{Semantic vs. syntactic}

\footnote{based on slides from Cambridge University}


There are two kinds of variable liveness : Semantic liveness and Syntactic liveness.


A variable x is \textbf{semantically} live at a node n if there is
some execution sequence starting at n whose (externally
observable) behaviour can be affected by changing the
value of x. Semantic liveness is concerned with
the execution behaviour of the program.

A variable is \textbf{semantically} live at a node if there is a
path to the exit of the flow graph along which its
value may be used before it is redefined. Syntactic liveness is concerned with properties of
the syntactic structure of the program.


So what is the difference between Semantic liveness and Syntactic liveness? syntactic liveness
is a computable approximation of semantic liveness.


Consider the example 


\begin{lstlisting}[language=C,frame=single, caption=An ,label = lst:expr2]
    int t = x * y;
    if ((x+1)*(x+1) == y) {
     t = 1;
    }
    if (x*x + 2*x + 1 != y) {
     t = 2;
    }
    return t;
\end{lstlisting}

In fact, t is dead in node \texttt{int t = x;} because one of the conditions will be true, 
so on every execution path t is redefined before it is returned.
The value assigned by the first instruction is never used.


But on read path from \ref{fig:liveex} through the
flowgraph, t is not
redefined before it's used,
so t is syntactically live at
the first instruction.Note that this path never
actually occurs during
execution.

\begin{figure}[h]
    \centering
    \includegraphics[width=0.3\textwidth]{liveex.png}
    \caption{CFG}
    \label{fig:liveex}
\end{figure}





\newpage

\section{Reaching Definitions}

The Reaching Definitions Problem is a data-flow problem used to answer the
following questions: Which definitions of a variable \textit{X} reach a given use of \textit{X} in
an expression? Is \textit{X} used anywhere before it is defined? A definition\textit{d} reaches a point \textit{p} if there exists path
from the point immediately following \textit{d} to \textit{p} such that \textit{d} is not killed(overwritten) along that path.



\subsection{Iterative   Algorithm}

Here is the iterative  algorithm.



\begin{algorithm}
	\caption{Reaching Defintions:Iterative Algorithm}\label{alg:reachingdefiterative}
	\hspace*{\algorithmicindent} \textbf{Input: control flow graph CFG = (N, E, Entry, Exit) } \\


	\begin{algorithmic}

		\State out[Entry] = $\emptyset$ \algorithmiccomment{Boundary condition}

		\For{\texttt{each basic block B other than Entry}}
		\State \texttt{out[B] = $\emptyset$} \algorithmiccomment{Initialization for iterative algorithm }
		\EndFor
		\While{Changes to any out[] occur}
		\For{\texttt{each basic block B other than Entry}}
		\State \texttt{$in[B] =  \cup (out[p])$, for all predecessors p of B}
		\State \texttt{$out[B] = f_B(in[B])$} \algorithmiccomment{$out[B]=gen[B]\cup (in[B]-kill[B]) $ }
		\EndFor

		\EndWhile
	\end{algorithmic}
\end{algorithm}




\subsection{Worklist   Algorithm}

\begin{algorithm}
	\caption{Reaching Defintions:Worklist Algorithm}\label{alg:reachingdefiterative}
	\hspace*{\algorithmicindent} \textbf{Input: control flow graph CFG = (N, E, Entry, Exit) } \\


	\begin{algorithmic}

		\State out[Entry] = $\emptyset$ \algorithmiccomment{Boundary condition}
		\State \textcolor{blue}{ChangedNodes = N}
		\For{\texttt{each basic block B other than Entry}}
		\State \texttt{out[B] = $\emptyset$} \algorithmiccomment{Initialization for iterative algorithm }
		\EndFor
		\While{ChangedNodes $\neq \emptyset$}
		\State \textcolor{blue}{Remove i from ChangedNodes}
		\State $in[B] =  \cup (out[p])$, for all predecessors p of B
		\State \textcolor{blue}{$oldout = out[i]$}
		\State $out[i] = f_i(in[i])$ \algorithmiccomment{$out[i]=gen[i]\cup (in[i]-kill[i]) $ }
		\If {\textcolor{blue}{oldout} $\neq out[i]$}

		\For{\texttt{all \textcolor{blue}{successors s of i}}}
		\State \textcolor{blue}{add s to ChangedNodes}
		\EndFor
		\EndIf

		\EndWhile
	\end{algorithmic}
\end{algorithm}



\subsection{Example}
Here comes an example of reaching definition.

\begin{figure}[!htb]
	\minipage{0.32\textwidth}
	\includegraphics[width=\linewidth]{rdex1.jpg}
	\caption{Pass 1}\label{fig:awesome_image1}
	\endminipage\hfill
	\minipage{0.32\textwidth}
	\includegraphics[width=\linewidth]{rdex2.jpg}
	\caption{Pass 2}\label{fig:awesome_image2}
	\endminipage\hfill
	\minipage{0.32\textwidth}%
	\includegraphics[width=\linewidth]{rdex3.jpg}
	\caption{Pass 3}\label{fig:awesome_image3}
	\endminipage
\end{figure}


\subsection{Summary}

\begin{center}
	\begin{tabular}{|c|c|}
		\hline Direction                         & Forward	\\
		\hline Domain                            & Sets	of	definitions                                    \\
		\hline Meet operator                     & \( \cup \)                                          \\
		\hline Top(T)                            & $\phi$                                              \\
		\hline Bottom                            & Universal Set                                       \\
		\hline Boundary condition                & $\mathrm{OUT[ENTRY]} = \phi$                          \\
		\hline Initialization for internal nodes & $\mathrm{OUT[B]} = \phi$                             \\
		\hline Finited escending chain?          & \checkmark                                          \\
		\hline Transfer function                 & $f_b(x) = \mathrm{Gen}_b \cup (x - \mathrm{Kill}_b)$ \\
		\hline Monotone\&Distributive?           & \checkmark                                          \\
		\hline
	\end{tabular}
\end{center}

\newpage

\section{Available Expressions Analysis}

\subsection{Motivation}

Programs may contain code whose result is needed, but in which some computation is simply a redundant
repetition of earlier computation within the same program. The concept of expression availability is useful in dealing with this situation.


\subsection{Backgroud Knowledge}

Any given program contains a finite number of expressions (i.e. computations which potentially
produce values),so we may talk about the set of all expressions of a program. Consider the program in
\ref{lst:expression1}




\begin{lstlisting}[language=C,frame=single, caption=An simple example containing some expressions ,label = lst:expression1]
    int z = x * y; 
    print s + t; 
    int w = u / v;
\end{lstlisting}


This program contian expression \texttt{x*y,s+t,u/v}.



\subsection{Problem Formulation}


Availability is a data-flow property of expressions: “Has the value of this expression already been computed?”
At each instruction, each expression in the programis either available or unavailable. So each instruction(or node of the flowgraph) has
an associated set of available expression.



\subsection{Semantic vs. Syntactic}

An expression is \textit{semantically} available at a node n if its value gets computed
(and not subsequently invalidated) along every execution sequence ending at n.

\begin{figure}[!htb]
	\minipage{0.5\textwidth}
	\includegraphics[width=\linewidth]{p1.png}
	\caption{Available expression example}\label{fig:p1}
	\endminipage\hfill
	\minipage{0.5\textwidth}
	\includegraphics[width=\linewidth]{p2.png}
	\caption{unavailable expression example}\label{fig:p2}
	\endminipage
\end{figure}


An expression is \textit{syntactically} available at a node n if its value gets computed
(and not subsequently invalidated) along every path from the entry of the flowgraph to n.


\begin{figure}[!htb]
	\minipage{0.5\textwidth}
	\includegraphics[width=\linewidth]{p4.png}
	\caption{x+y is semantically available}\label{fig:p4}
	\endminipage\hfill
	\minipage{0.4\textwidth}
	\includegraphics[width=\linewidth]{p3.png}
	\caption{x+y is syntactically unavailable}\label{fig:p3}
	\endminipage
\end{figure}


On the path in red from Figure \ref{fig:p3} through the flowgraph, \(x+y\) is only
computed once, so \(x+y\) is syntactically unavailable at the last instruction.


Whereas with live variable analysis we found safety in assuming that
more variables were live, here we find safety in assuming that fewer
expressions are available. Because if an expression is deemed to be available, we
may do something dangerous (e.g. remove an instruction which recomputes its value).
So sometimes safe means more, but sometimes means less.

\begin{figure}[H]
	\minipage{0.5\textwidth}
	\includegraphics[width=\linewidth]{p5.png}
	\caption{Semantic vs. syntactic}\label{fig:p5}
	\endminipage\hfill
	\minipage{0.5\textwidth}
	\includegraphics[width=\linewidth]{p6.png}
	\caption{Semantic vs. syntactic}\label{fig:p6}
	\endminipage
\end{figure}


\subsection{Summary}


\begin{center}
	\begin{tabular}{|c|c|}
		\hline Direction                         & Forward                                           \\
		\hline Domain                            & Sets of expressions                                    \\
		\hline Meet operator                     & \( \cap \)                                          \\
		\hline Top(T)                            & Universal Set                                             \\
		\hline Bottom                            & $\phi$                                     \\
		\hline Boundary condition                & $\mathrm{OUT[ENTRY]} = \phi$                          \\
		\hline Initialization for internal nodes & $\mathrm{OUT[B]} = T$                             \\
		\hline Finited escending chain?          & \checkmark                                          \\
		\hline Transfer function                 & $f_b(x) = \mathrm{Gen}_b \cup (x - \mathrm{Kill}_b)$ \\
		\hline Monotone\&Distributive?           & \checkmark                                          \\
		\hline $\mathrm{Kill}_b$ & all E such that block b defines a variable in E \\
    \hline $\mathrm{Gen}_b$ & all E such that block b evaluates E and doesn’t later kill it \\
    \hline
	\end{tabular}
\end{center}
\section{Foundations of Data Flow Analysis}



We saw a lot of examples of data flow analysis, eg. reaching definitions etc. Although 
there were differences between differeent types of data flow analysis, they did share number of 
things in common. Our goal is to develop a general purpose data flow analysis.




































Having shown several useful examples of the data-flow abstraction, 
we now study the family of data-flow schemas as a whole, abstractly. 
We shall answer several basic questions about data-flow algorithms formally:

\begin{itemize}

\item Under what circumstances is the iterative algorithm used in data-flow analysis correct?
\item How precise is the solution obtained by the iterative algorithm?
\item Will the iterative algorithm converge?
\item How fast is the convergence?
\end{itemize}


\subsection{Partial Order}\footnote{Based on \url{https://pages.cs.wisc.edu/~horwitz/CS704-NOTES/2.DATAFLOW.html}}

A binary relation R on a set S is called a partial ordering(poset), or partial order if and only if it is:

\begin{itemize}
\item \textbf{Reflexive} \(x \leq x\)
\item \textbf{Antisymmetric} if \(x \leq y\) and \(y \leq x\) then \(x = y\)
\item \textbf{Transitive} if \(x \leq y\) and \(y \leq z\) then \(x \leq z\)
\end{itemize} 



\subsection{Lattices}

A lattice is a poset in which every pair of elements has:

\begin{itemize}
\item a Least Upper Bound (the join of the two elements), and
\item a Greatest Lower Bound (the meet of the two elements).
\end{itemize}    



\subsection{Complete lattices}


A complete lattice is a lattice in which all subsets have a greatest lower bound 
and a least upper bound (the bounds must be in the lattice, but not necessarily 
in the subsets themselves). Note that Every finite lattice (i.e., S is finite) is complete.


\subsection{Monotonic and distributive functions}

A function f: L → L (where L is a lattice) is monotonic iff for all x,y in L: x ⊆ y implies f(x) ⊆ f(y).

A function f: L → L (where L is a lattice) is distributive iff for all x,y in L: f(x meet y) = f(x) meet f(y).

Every distributive function is also monotonic (proving that could be good practice!) but not vice versa. For the GEN/KILL dataflow problems, all dataflow functions are distributive. For constant propagation, all functions are monotonic, but not all functions are distributive.


\subsection{Fixed points}

x is a fixed point of function f iff f(x) = x.

\subsection{Meet Operator}



\newpage



\section{Introduction to Static Single Assignment}

Many dataflow analysis need to find the use-sites of each defined variable or the definition-sites of each variable used in an expression.
 The \textit{def-use chain} is a data structure that makes this efficient: for each statement in the flow graph,
  the compiler can keep a list of pointers to all the use sites of variables defined there, and a list of pointers 
  to all definition sites of the variables used there. An improvement on the idea of \textit{def-use chains}
   is \textit{static single-assignment form}, or \textit{SSA form},\footnote{In SSA, each temporary has only one definition in program, for each use u
   of r, only one definition of r reaches u.} an intermediate representation in which each variable has only one definition in the program text. SSA is very useful for many optimizations such as Loop-Invariant Code Motion and Copy Propagation.

% \subsection{Motivation}

% \begin{itemize}
%     \item The values in reused locations may be provably independent.
%     \item It would be nice if we could traverse directly between related uses and def's
% \end{itemize}

\subsection{Definition-Use and Use-Definition Chains}


\begin{definition}{Use-Definition (UD) Chains}
	For a given definition of a variable X, what are all of its uses?

\end{definition}



\begin{definition}{Definition-Use (DU) Chains}
	For a given use of a variable X, what are all of the reaching definitions of X?

\end{definition}



Unfortunately, it is expensive to use UD and DU chains, because if we have $N$ defs, and $M$ uses,
 the space complexity is $O(NM)$. An example is in Figure \ref{fig:p38}


\begin{figure}[htb]
	\centering
	\includegraphics[width=0.3\textwidth]{p38.png}
	\caption{If a variable has $N$ uses and $M$ definitions (which occupy about $N + M$ instructions in a program), it takes space (and time) proportional to $N · M$ to represent def-use chains – a quadratic blowup.}
	\label{fig:p38}
\end{figure}


\subsection{Static Single Assignment(SSA)}

\begin{definition}{Static Single Assignment }
	Static Single Assignment is an IR where every variable is assigned a value at most once in the program text.
\end{definition}








\begin{definition}{the $\Phi$ function}

	$\Phi$ merges multiple definitions along multiple control paths into a single definition.

	At a basic block with p predecessors, there are p arguments to the $\Phi$ functions.

	$$ x_{\text {new }} \leftarrow \Phi\left(\mathbf{x}_1, \mathbf{x}_2, \mathbf{x}_3, \ldots, \mathbf{x}_{\mathrm{p}}\right)
	$$
\end{definition}

\subsubsection{Why SSA is useful?}

\textbf{ \large \textit{Useful for Dataflow Analysis}} Dataflow analysis and optimization algorithms can be made simpler when each variable has only one definition.

\textbf{ \large \textit{Less space and time complexity}} If a variable has N uses and M definitions (which occupy about N + M instructions in a program), it takes space (and time) proportional to N · M to represent def-use chains – a quadratic blowup. For almost all realistic programs, the size of the SSA form is linear in the size of the original program.


\textbf{ \large \textit{Simplify some algorithms}} Uses and defs of variables in SSA form relate in a useful way to the dominator structure of the control-flow graph, which simplifies algorithms such as interference-graph construction.


\textbf{ \large \textit{Eliminate needless relationship}} Unrelated uses of the same variable in the source program become different variables in SSA form, eliminating needless relationships shown in Listing\ref{exp:1}.

\begin{lstlisting}[label={exp:1},caption={No reason why both loops should be forced to use same register to hold index
	register. SSA renames second i to a new temporary which may lead to better register
	allocation/optimization.}]
for i <- 1 to N  do A[i] <- 0
for i <- 1 to M  do s <- s + B[i]
\end{lstlisting}


\subsection{How to represent SSA?}

In straight-line code, such as within a basic block, it is easy to see that each instruction can define a fresh new 
variable instead of redefining an old one shown in Figure \ref{fig:p42-43}


\begin{figure}[htb]
	\centering
	\begin{subfigure}{0.2\textwidth}
		\centering
		\includegraphics[width=\textwidth]{p42.png}
		\caption{A straight-line program.}
		\label{fig:p42}
	\end{subfigure}
	\begin{subfigure}{0.25\textwidth}
		\centering
		\includegraphics[width=\textwidth]{p43.png}
		\caption{The program in single-assignment form.}
		\label{fig:p43}
	\end{subfigure}
	\caption{SSA for straight-line code}
	\label{fig:p42-43}
\end{figure}


But when two control-flow paths merge together, it is not obvious how to have only one assignment for each variable. To solve this problem we introduce a notational fiction, called a $\Phi$ function. Figure \ref{fig:p44} shows that we can combine a1 (defined in block 1) and a2 (defined in block 3) using the function $a3 \leftarrow \Phi(a1, a2)$.


\begin{figure}[H]
	\centering
	\includegraphics[width=0.8\textwidth]{p44.png}
	\caption{(a) A program with a control-flow join; (b) the program transformed to single-assignment form; (c) edge-split SSA form.}
	\label{fig:p44}
\end{figure}


Unlike ordinary mathematical functions, $\Phi$(a1, a2) yields a1 if control reaches block 4 along the edge $2 \rightarrow 4$, and yields a2 if control comes in on edge $3 \rightarrow 4$.


\subsubsection{How does the $\phi$-function know which edge was taken?}


If we must execute the program, or translate it to executable form, we can “implement” the $\Phi$-function using 
a move instruction on each incoming edge as shown in Figure \ref{fig:p39-40}. However, in many cases, we simply need the connection of uses to definitions and don’t need to “execute” the $\Phi$-functions during optimization. In these cases, we can ignore the question of which value to produce.

\begin{figure}[H]
	\centering
	\begin{subfigure}{0.3\textwidth}
		\centering
		\includegraphics[width=\textwidth]{p39.png}
		\caption{Original code}
		\label{fig:p39}
	\end{subfigure}
	\begin{subfigure}{0.3\textwidth}
		\centering
		\includegraphics[width=\textwidth]{p40.png}
		\caption{Code after moving instruction.}
		\label{fig:p40}
	\end{subfigure}
	\caption{Implementing $\Phi$-function}
	\label{fig:p39-40}
\end{figure}




\subsection{Converting to SSA form}

The algorithm for converting a program to SSA form is roughly as follows:

\begin{itemize}
	\item 1. adds $\Phi$ functions for the variables, and then
	\item 2. renames all the definitions and uses of variables using subscripts.
\end{itemize}




The challenge to covert CFG to SSA is where to insert $\Phi$-functions so that every use has exactly one def.
Once this property has been achieved, the resulting webs can be renamed (e.g., by adding subscripts to
their variable name) accordingly.

A simple-minded approach is just to insert $\Phi$-functions for every variable at the head of every basic
block with more than one predecessor. But this creates a more complex CFG than necessary, with extra
variables that slow down optimization.

When does basic block \texttt{z} truly need to have the $\Phi$-function for a variable a; that is, inserting a = $\Phi$(a, a)
at its beginning? This question can be answered using the path convergence criterion: the $\Phi$(a, a) is needed
when:


\begin{itemize}
\item There exist two nodes \texttt{x} and \texttt{y} that define variable a.
\item There are nonempty paths from x and y to z that are disjoint except at the final z node. Along these
two paths, a is defined only at x and y.
\end{itemize}


This rule implies that \texttt{z} must be a node with multiple predecessors, because otherwise two paths would
have to share the single predecessor and therefore would not be disjoint. It similarly implies that \texttt{z} can also
appear in the middle of one of the paths from \texttt{x} and \texttt{y} to \texttt{z}, but cannot be in the middle of both.
Note that for evaluating the path convergence criterion, we consider the start node of the CFG to implicitly define every variable, 
representing its initial value (initialized or uninitialized) on entry.

Although path convergence gives us a clear criterion for when to insert a $\Phi$-function, it is expensive to
evaluate directly. SSA conversion is therefore usually done using a dominator analysis.



\subsubsection{Trivial SSA}

Trivial SSA form is based on a simple observation: $\Phi$ functions are only needed for variables that are "live" after the $\Phi$ function.

\begin{itemize}
	\item Each assignment generates a fresh variable.
	\item At each join point insert $\Phi$ for all live variables.
\end{itemize}


Trivial SSA will generate some useless $\Phi$ functions. An example is shown in Figure \ref{fig:p41}. 
So a $\Phi$-function is not needed for every variable at each point.

\begin{figure}[H]
	\centering
	\includegraphics[width=0.5\textwidth]{p41.png}
	\caption{$x2 \leftarrow \Phi(x1,x1)$ is useless because $x2$ is equal to $x1$.}
	\label{fig:p41}

\end{figure}



\subsubsection{Minimal SSA}
Minimal SSA is an updated version compared to trivial SSA. This means:

\begin{itemize}
	\item Each assignment generates a fresh variable.
	\item At each join point insert $\Phi$ for all live variables with multiple outstanding defs.
\end{itemize}

\subsubsection{Path-convergence criterion}

There should be a $\Phi$-function for variable b at node z of the flow graph
exactly when all of the following are true:

\begin{itemize}
	\item 1. There is a block x containing a definition of b shown in Figure \ref{fig:p235},
	\item 2. There is a block y (with $y \neq x$) containing a definition of b shown in Figure \ref{fig:p236},
	\item 3. There is a nonempty path $P_{xz}$ of edges from x to z shown in Figure \ref{fig:p237},
	\item 4. There is a nonempty path $P_{yz}$ of edges from y to z shown in Figure \ref{fig:p238},
	\item 5. Paths $P_{xz}$ and $P_{yz}$ do not have any node in common other than z  shown in Figure \ref{fig:p239}, and
	\item 6. The node z does not appear within both $P_{xz}$ and $P_{yz}$ prior to the
	      end, though it may appear in one or the other shown in Figure \ref{fig:p240}.
\end{itemize}


Show that the phrase “though it may appear in one or the other” in the last statement is necessary



% \begin{figure}[H]
% 	\centering
% 	\includegraphics[width=0.3\textwidth]{p235.png}
% 	\caption{}
% 	\label{fig:p235}
% \end{figure}



% \begin{figure}[H]
% 	\centering
% 	\includegraphics[width=0.3\textwidth]{p236.png}
% 	\caption{}
% 	\label{fig:p236}
% \end{figure}


% \begin{figure}[H]
% 	\centering
% 	\includegraphics[width=0.3\textwidth]{p237.png}
% 	\caption{}
% 	\label{fig:p237}
% \end{figure}



\begin{figure}[H]
	\minipage{0.32\textwidth}
	  \includegraphics[width=\linewidth]{p235.png}
	  \caption{}\label{fig:p235}
	\endminipage\hfill
	\minipage{0.32\textwidth}
	  \includegraphics[width=\linewidth]{p236.png}
	  \caption{}\label{fig:p236}
	\endminipage\hfill
	\minipage{0.32\textwidth}%
	  \includegraphics[width=\linewidth]{p237.png}
	  \caption{}\label{fig:p237}
	\endminipage
\end{figure}




\begin{figure}[H]
	\minipage{0.32\textwidth}
	  \includegraphics[width=\linewidth]{p238.png}
	  \caption{}\label{fig:p238}
	\endminipage\hfill
	\minipage{0.32\textwidth}
	  \includegraphics[width=\linewidth]{p239.png}
	  \caption{}\label{fig:p239}
	\endminipage\hfill
	\minipage{0.32\textwidth}%
	  \includegraphics[width=\linewidth]{p240.png}
	  \caption{}\label{fig:p240}
	\endminipage
\end{figure}


% \begin{figure}[H]
% 	\centering
% 	\includegraphics[width=0.3\textwidth]{p238.png}
% 	\caption{}
% 	\label{fig:p238}
% \end{figure}


% \begin{figure}[H]
% 	\centering
% 	\includegraphics[width=0.3\textwidth]{p239.png}
% 	\caption{}
% 	\label{fig:p239}
% \end{figure}


% \begin{figure}[H]
% 	\centering
% 	\includegraphics[width=0.3\textwidth]{p240.png}
% 	\caption{}
% 	\label{fig:p240}
% \end{figure}


We consider the start node to contain an implicit definition of every variable, either because the variable may be a formal parameter or to represent the notion of a
$\leftarrow$ uninitialized without special cases. A $\Phi$-function itself counts as a definition of a, so the path-convergence criterion must be considered as a set of equations to be satisfied. 
As usual, we can solve them by iteration as shown in Algorithm \ref{alg:Iterated path-convergence criterion}.


\begin{algorithm}
	\caption{Iterated path-convergence criterion}\label{alg:Iterated path-convergence criterion}
	\begin{algorithmic}

		\While{there are nodes $x, y, z$ satisfying conditions 1–5
			and \\ $z$ does not contain a $\Phi$-function for a}
		\State  insert a $\leftarrow$ $\Phi$(a, a, . . . , a) at node Z
		\EndWhile
	\end{algorithmic}
\end{algorithm}

\subsubsection{Dominance property of SSA form}

The iterated path-convergence algorithm for placing $\Phi$-functions is not practical, since it would be very costly to examine every triple of nodes x, y, z, and every path leading from x and y.  A much more efficient algorithm using the dominator tree of the flow graph as shown in Figure \ref{fig:p45}.


\begin{figure}[H]
	\centering
	\includegraphics[width=\textwidth]{p45.png}
	\caption{ Conversion of a program to static single-assignment form. Node 7 is a postbody node, inserted to make sure there is only one loop edge; such nodes are not strictly necessary but are sometimes helpful.}
	\label{fig:p45}

\end{figure}


\begin{definition}{Strictly dominance}
	x strictly dominates w (x sdom w) iff impossible to reach w without passing through x first.
\end{definition}

\begin{definition}{Dominance}
	x  dominates w (x dom w) iff x sdom w or x = w.
	$$
		\operatorname{Dom}(n)= \begin{cases}\{n\} & \text { if } n=n_0 \\ \{n\} \cup\left(\bigcap_{p \in \operatorname{preds}(n)} \operatorname{Dom}(p)\right) & \text { if } n \neq n_0\end{cases}
	$$
\end{definition}

\begin{definition}{Dominance tree}
	x sdom w iff x is a proper ancestor of w.

\end{definition}

\begin{definition}{Dominance Frontier}


	The dominance frontier of a node x is the set of all nodes w such that
	x dominates a predecessor of w, but does not strictly dominate w.

	$$
		F(x)=  \{w | \texttt{x  dom pred(w) AND   !(x  sdom  w) } \}
	$$
\end{definition}





An essential property of static single assignment form is that definitions dominate uses; more specifically,
\begin{itemize}
	\item  If x is the ith argument of a $\Phi$-function in block n, then the definition of x dominates the ith predecessor of n.
	\item  If x is used in a non-$\Phi$ statement in block n, then the definition of x dominates n
\end{itemize}

\begin{note}{Dominance Property of SSA	}
	In SSA,

	\begin{itemize}
		\item If x i is used in $x \leftarrow \Phi (..., x_i , ...)$, then $BB(x_i )$ dominates ith predecessor of $BB(\Phi)$
		\item If x is used in $y \leftarrow ...x...$,then BB(x) dominates BB(y)
	\end{itemize}
\end{note}


\textbf{ \large \textit{Dominance frontier criterion.}} Whenever node x contains a definition of some variable a, then any node z in the dominance frontier of x needs a $\Phi$-function for a.


\textbf{ \large \textit{Iterated dominance frontier.}} Since a $\Phi$-function itself is a kind of definition, we must iterate the dominance-frontier criterion until there are no nodes that need $\Phi$-functions.


\textbf{ \large \textit{Theorem.}} The iterated dominance frontier criterion and the iterated path convergence criterion specify exactly the same set of nodes at which to put $\Phi$-functions

\begin{figure}[htb]
	\centering
	\includegraphics[width=\textwidth]{p46.png}
	\caption{Node 5 dominates all the nodes in the grey area. (a) Dominance frontier of node 5 includes the nodes (4, 5, 12, 13) that are targets of edges crossing from the region dominated by 5 (grey area including node 5) to the region not strictly dominated by 5 (white area including node 5). (b) Any node in the dominance frontier of n is also a point of convergence of nonintersecting paths, one from n and one from the root node. (c) Another example of converging paths $P_{1,5}$ and $P_{5,5}$.}
	\label{fig:p46}

\end{figure}


\begin{proof}{Proof}
	The sketch of a proof that shows if w is in the dominance frontier of a definition, then it must be a point of convergence.

	Suppose there is a definition of variable a at some node n (such as node 5 in Figure  \ref{fig:p46}b), and node w (such as node 12 in Figure  \ref{fig:p46}b) is in the dominance frontier of n. The root node implicitly contains a definition of every variable, including a. There is a path $P_{rw}$ from the root node (node 1 in Figure \ref{fig:p46}) to w that does not go through n or through any node that n dominates; and there is a path $P_{nw}$ from n to w that goes only through dominated nodes. These paths have w as their first point of convergence.
\end{proof}


\subsection{Computing the dominance frontier}

To insert all the necessary $\Phi$-functions, for every node n in the flow graph we need DF[n], its dominance frontier. Given the dominator tree, we can efficiently compute the dominance frontiers of all the nodes of the flow graph in one pass. We define two auxiliary sets

\begin{itemize}
	\item $DF_{local}[n]$ The successors of n that are not strictly dominated by n;
	\item $DF_{up}[n]$ Nodes in the dominance frontier of n that are not dominated by n’s immediate dominator.
\end{itemize}

The dominance frontier of n can be computed from $DF_{local}[n]$  and $DF_{up}[n]$


$$
	D F[n]=D F_{\text {local }}[n] \quad \cup \quad \bigcup_{c \in \text { children }[n]} D F_{\text {up }}[c]
$$

where children[n] are the nodes whose immediate dominator (idom) is n.


To compute $DF_{local}[n]$ \ref{alg:computeDF} more easily (using immediate dominators instead of dominators), we use the following theorem: $DF_{local}[n]$ = the set of those successors of n whose immediate dominator is not n. The following computeDF function should be called on the root of the dominator tree (the start node of the flow graph). It walks the tree computing DF[n] for every node n: it computes $DF_{local}[n]$ by examining the successors of n, then combines $DF_{local}[n]$ and (for each child c) $DF_{up}[n]$.a

\begin{algorithm}
	\caption{computeDF}\label{alg:computeDF}
	\begin{algorithmic}

		\State $S \gets \{\}$
		\For{each node y in succ[n]} \Comment{This loop computes $DF_{local}[n]$}
		\If {idom(y) $\neq$ n}
		\State $S\leftarrow S \cup \{y\}$
		\EndIf
		\EndFor

		\For{each child c of n in the dominator tree}
		\State computeDF[c]
		\For{each element w of DF[c]} \Comment{This loop computes $DF_{up}[n]$}
		\If {n does not dominate w}
		\State $S\leftarrow S \cup \{w\}$
		\EndIf
		\EndFor
		\EndFor
	\end{algorithmic}
\end{algorithm}

This algorithm is quite efficient. It does work proportional to the size (number of edges) of the original graph, plus the size of the dominance frontiers it computes. Although there are pathological graphs in which most of the nodes have very large dominance frontiers, in most cases the total size of all the DFs is approximately linear in the size of the graph, so this algorithm runs in “practically” linear time.




\subsection{Inserting $\Phi$-functions}

Starting with a program not in SSA form, we need to insert just enough $\Phi$-functions to satisfy the iterated dominance frontier criterion. To avoid re-examining nodes where no $\Phi$-function has been inserted, we use a work-list algorithm.

\begin{algorithm}[H]
	\caption{Place-$\Phi$-Functions}\label{alg:Place-Phi-Functions}
	\begin{algorithmic}
		\For{each node n}
		\For{each variable a in $A_{orig}[n]$}
		\State defsites[a] $\gets$ defsites[a] $\cup \{n\}$
		\EndFor
		\EndFor


		\For{each variable a}
		\State W $\gets$ defsites[a]
		\While{ W not empty}
		\State{remove some node n from W}
		\For{each y in DF[n]}
		\If{y $\not\in$ $A_{\Phi}[a]$}
		\State{insert the statement a $\gets$ $\Phi$(a, a, . . . , a) at the top of block y, where the $\Phi$-function has as many arguments as y has predecessors}
		\State{$A_{\Phi}[a] \gets A_{\Phi}[a] \cup \{ y\}$}
		\If{a $\not\in$ $A_{orig}[y]$}
		\State{$W \gets W \cup \{y \}$}
		\EndIf

		\EndIf
		\EndFor
		\EndWhile
		\EndFor

	\end{algorithmic}
\end{algorithm}


Algorithm\ref{alg:Place-Phi-Functions} starts with a set V of variables, a graph G of controlflow nodes – each node is a basic block of statements – and for each node n a set $A_{orig}$[n] of variables defined in node n. The algorithm
computes $A_{\Phi}[a]$, the set of nodes that must have $\Phi$-functions for variable
a. Sometimes a node may contain both an ordinary definition and a
$\Phi$-function for the same variable; for example, in Figure \ref{fig:p46}b, a $\in$ $A_{orig}[2]$
and 2 $\in$ $A_{\Phi}[a]$.


This algorithm does a constant amount of work (a) for each node and edge in the control-flow graph, (b) for each statement in the program, (c) for each element of every dominance frontier, and (d) for each inserted $\Phi$-function. For a program of size N, the amounts a and b are proportional to N, c is usually approximately linear in N. The number of inserted $\Phi$-functions (d) could be $N^2$ in the worst case, but empirical measurement has shown that it is usually proportional to N. So in practice, Algorithm \ref{alg:Place-Phi-Functions} runs in approximately linear time.


\subsection{Renaming the variables}

After the $\Phi$-functions are placed, we can walk the dominator tree, renaming the different definitions (including $\Phi$-functions) of variable a to a1, a2, a3 and so on. Rename each use of a to use the closest definition d of a that is above a in the dominator tree. Algorithm  renames all uses and definitions of variables, after the $\Phi$-functions have been inserted by Algorithm \ref{alg:Renaming variables}. In traversing the dominator tree, the algorithm “remembers” for each variable the most recently defined version of each variable, on a separate stack for each variable. Although the algorithm follows the structure of the dominator tree – not the flow graph – at each node in the tree it examines all outgoing flow edges, to see if there are any $\Phi$-functions whose operands need to be properly numbered.



\begin{algorithm}[H]
	\caption{Renaming variables.}\label{alg:Renaming variables}
	\begin{algorithmic}
		\State Initialization:

		\For{each variable a}
		\State{Count[a] $\gets$ 0}
		\State{Stack[a] $\gets$ empty}
		\State{push 0 onto Stack[a]}
		\EndFor

		\Function{Rename(n) }{}
		\For{each statement S in block n}
		\If{S is not a $\Phi$-function}
		\For{each use of some variable x in S}
		\State{i $\gets$ top(Stack[x])}
		\State{replace the use of x with $x_i$ in S}
		\EndFor
		\EndIf
		\For{each definition of some variable a in S}

		\State Count[a] $\gets$ Count[a]+1
		\State i $\gets$ Count[a]
		\State push i onto Stack[a]
		\State replace definition of a with definition of $a_i$ in S
		\EndFor
		\EndFor

		\For{each successor Y of block n,}
		\State Suppose n is the jth predecessor of Y
		\For{each $\Phi$-function in Y}
		\State suppose the jth operand of the $\Phi$-function is a
		\State i $\gets$ top(Stack[a])
		\State replace the jth operand with $a_i$

		\EndFor
		\EndFor
		\For{each child X of n}
		\State Rename(X)
		\EndFor
		\For{each definition of some variable a in the original S}
		\State pop Stack[a]
		\EndFor
		\EndFunction
	\end{algorithmic}
\end{algorithm}

\subsubsection{Example}




\includepdf[pages={1-}]{p53.pdf}

\subsection{Edge Splitting}

Some analyses and transformations including reverse transformation from SSA back into a normal form are simpler if there is never a controlflow edge that leads from a node with multiple successors to a node with multiple predecessors. To give the graph this unique successor or predecessor property, we perform the following transformation: For each control-flow edge a $\gets$ b such that a has more than one successor and b has more than one predecessor, we create a new, empty controlflow node z, and replace the a $\gets$ b edge with an a $\gets$ z edge and a z $\gets$ b edge.

An SSA graph with this property is in edge-split SSA form. Figure \ref{fig:p44} illustrates edge splitting. Edge splitting may be done before or after insertion of $\Phi$-functions.
\newpage

\section{SSA-Style optimizations}

\subsection{Constant Propagation}
\begin{note}{notes}
	\begin{itemize}
		\item If  v $\gets$ c , replace all uses of v with c
		\item If  v $\gets$  $\Phi$ (c,c,c)  (each input is the same constant), replace all uses of v with c
	\end{itemize}
\end{note}


\begin{algorithm}
	\caption{SSA-CP}\label{alg:SSA-CP}
	\begin{algorithmic}
		\State{W $\gets$ list of all defs}
		\While{!W.isEmpty()}

		\State{Stmt S $\gets$ W.removeOne()}
		\If{(S has form v $\gets$ c) or
			(S has form v $\gets$ $\Phi$(c,...,c))}
		\State delete S
		\For{each stmt U that uses v}
		\State {replace v with c in U}
		\State {W.add(U)}
		\EndFor
		\EndIf
		\EndWhile

	\end{algorithmic}
\end{algorithm}

\subsection{Conditional Constant Propagation }

In this section, we want to talk something about Conditional Constant Propagation. 
Before we talk about it, we need to know something about the history of Constant Propagation.
There are four algorithms for determining constants.
 They are described in order of increasing power; each algorithm finds at least the
constants found by the previous algorithm. These algorithms are among the
simplest, fastest, and most powerful global constant propagation algorithms
known. The first three algorithms are reformulations of the work of others;
the fourth is new and contains the best features of each of the previous three.
Figure \ref{fig:p223} shows the relationship among the four algorithms.


\begin{figure}[H]
	\centering
	\includegraphics[width=0.5\textwidth]{p223.png}
	\caption{Relationship among the four constant propagation algorithms}
	\label{fig:p223}

\end{figure}


The first algorithm, Simple Constant (SC), was developed by Kildall\cite{kildall1973unified}. 
Kildall was among the first to describe the
constant propagation problem and to give an algorithmic solution.



The second algorithm, Sparse Simple Constant (SSC), is an easily understood reformulation of an algorithm developed by Reif and Lewis\cite{reif1977symbolic}. 
This algorithm uses a data
structure called the static single assignment graph (SSA graph). The SSA
graph is a variant of the global value graph of Reif and Lewis, which in
turn is based on the p-graph of Shapiro and Saint. The SSA graph allows
this algorithm to find a class of constants equivalent to those of SC, yet the
algorithm is faster than SC by a factor proportional to the number of
variables in the program. Indeed, the speedup can be proportional to the
product of the number of variables in the program and the number of edges
in the program flow graph. It is unfortunate that this algorithm was not
recognized for many years, since it works in time linear in the size of the SSA
graph.


The third algorithm, Conditional Constant (CC), is a variant of Wegbreit’s
Algorithm\cite{wegbreit1975property}. CC discovers all constants
that can be found by evaluating all conditional branches with all constant
operands, but it uses the same input data structures and is asymptotically as
slow as SC. The attraction of CC is that it propagates the values in such a
way that when conditional branches are found to have a constant conditional
expression, the search for constants can ignore parts of the program that are
never executed. The algorithm does unreachable code elimination in combination with constant propagation. 
The first benefit of this approach is that
the algorithm may run faster than SC, since it need not evaluate the sections
of the program that are never executed. A second benefit is that values
created in the unreachable areas cannot possibly kill potential constants, and
thus CC can find more constants than can SC.

The fourth algorithm, Sparse Conditional Constant (SCC) finds the same class of constants as CC, yet has
the same speedup over CC as SSC has over SC.


Wegman and Zadeck's Sparse Conditional Constant (SCC) algorithm was used to find constant expressions, 
constant conditions, and unreachable code \cite{wegman1991constant}. The output of the SCC algorithm is 
an association of variables to one of $\lbrace \bot, c, \top \rbrace$, where $\bot$ marks a variable that
 can hold different values at different times, and $\top$ means the variable is not executed. In addition, 
 every flow-graph node (corresponding to a quadruple) is marked as executable or non-executable. 
 We then walk the flow-graph, eliminating dead-code (quadruples marked non-executable), replacing 
 constant variables with their values, and changing constant conditional branches to goto statements.

\begin{note}{notes}
	\begin{itemize}
		\item Assume all blocks unexecuted until proven otherwise
		\item Assume all variables are not executed (only with proof of assignment of a non-constant value do we assume not constant)
	\end{itemize}
\end{note}

\subsubsection{Example}

\begin{figure}[H]
	\centering
	\includegraphics[width=\textwidth]{p47.pdf}
	\caption{Original code. The black block is marked as unexecuted}
	\label{fig:p47}

\end{figure}




\begin{figure}[H]
	\centering
	\includegraphics[width=0.8\textwidth]{p48.pdf}
	\caption{The read block is marked as executed. After walking the first two blocks, the value is shown above.}
	\label{fig:p48}

\end{figure}



\begin{figure}[H]
	\centering
	\includegraphics[width=0.8\textwidth]{p49.pdf}
	\caption{After walking 5 blocks.}
	\label{fig:p49}

\end{figure}



\begin{figure}[H]
	\centering
	\includegraphics[width=0.8\textwidth]{p50.pdf}
	\caption{Now k2 is $\bot$, so the \texttt{return j2} is reachable.}
	\label{fig:p50}

\end{figure}



\begin{figure}[H]
	\centering
	\includegraphics[width=0.5\textwidth]{p51.pdf}
	\caption{Code after applied SCC.}
	\label{fig:p51}

\end{figure}


% \begin{figure}[!b]
%      \centering
%      \begin{subfigure}{0.45\textwidth}
%      \centering
%          \includegraphics[width=\textwidth]{p47.pdf}
%          \caption{Original code}
%          \label{fig:p47}
%      \end{subfigure}
%      \begin{subfigure}{0.6\textwidth}
%      \centering
%          \includegraphics[width=\textwidth]{p48.pdf}
%          \caption{Code after moving instruction.}
%          \label{fig:p48}
%      \end{subfigure}
%           \begin{subfigure}{0.6\textwidth}
%      \centering
%          \includegraphics[width=\textwidth]{p49.pdf}
%          \caption{Code after moving instruction.}
%          \label{fig:p49}
%      \end{subfigure}
%           \begin{subfigure}{0.6\textwidth}
%      \centering
%          \includegraphics[width=\textwidth]{p50.pdf}
%          \caption{Code after moving instruction.}
%          \label{fig:p50}
%      \end{subfigure}
%           \begin{subfigure}{0.6\textwidth}
%      \centering
%          \includegraphics[width=\textwidth]{p51.pdf}
%          \caption{Code after moving instruction.}
%          \label{fig:p51}
%      \end{subfigure}
%         \caption{Implementing $\Phi$-function}
%         \label{fig:p47-51}
% \end{figure}




\subsection{Copy Propogation}

\begin{note}{notes}
	\begin{itemize}
		\item  delete x $\gets \Phi$ (y,y,y) and replace all x with y
		\item delete x $\gets$ y and replace all x with y
	\end{itemize}
\end{note}



\subsection{Aggressive Dead Code Elimination}

We can easily define the standard algorithm \ref{alg:Dead Code Elimination} below, but this algorithm may leave zombies.
Look at the example in Figure \ref{fig:p55-56}


\begin{algorithm}[H]
	\caption{Dead Code Elimination}\label{alg:Dead Code Elimination}
	\begin{algorithmic}
		\State {W $\gets$  list of all defs}
		\While{!W.isEmpty}
		\State{Stmt S $\gets$ W.removeOne}
		\If{$|S.users| != 0$}
		\State{\textbf{continue}}
		\EndIf
		\If{S.hasSideEffects()}
		\State{\textbf{continue}}
		\EndIf
		\For {def in S.operands.definers}
		\State{def.users $\gets$ def.users - \{S\}}
		\If{$|def.users| == 0$}
		\State{W $\gets$ W UNION \{def\}}
		\EndIf
		\EndFor
		\State{delete S}
		\EndWhile
	\end{algorithmic}
\end{algorithm}


\begin{figure}[H]
	\centering
	\begin{subfigure}{0.4\textwidth}
		\centering
		\includegraphics[width=\textwidth]{p55.pdf}
		\caption{Original code. We can easily find that instructions relating to $i$ are dead and can be eliminated.}
		\label{fig:pp55}
	\end{subfigure}
	\begin{subfigure}{0.4\textwidth}
		\centering
		\includegraphics[width=\textwidth]{p56.pdf}
		\caption{SSA format code. Since there is a circle use chain so we can not remove instructions relating to $i$ because $i_1$ uses $i_2$, and $i_2$ uses $i_1$}.
		\label{fig:p56}
	\end{subfigure}


	\caption{An example to illustrate standard DCE can leave zombies.}
	\label{fig:p55-56}
\end{figure}

So instead of assuming everything is live until proven dead, we go another way: assuming everything is dead until proven live
shown in algorithm \ref{alg:Aggressive Dead Code Elimination}.



\begin{algorithm}[H]
	\caption{Aggressive Dead Code Elimination}\label{alg:Aggressive Dead Code Elimination}
	\begin{algorithmic}
		\Function{init}{}
		\State{mark as {\color{blue}live} all stmts that have side-effects:}
		\State{ \,\,\,\,\,\,\,\,    I/O}
		\State{  \,\,\,\,\,\,\,\,   stores into memory}
		\State{  \,\,\,\,\,\,\,\,   returns}
		\State{  \,\,\,\,\,\,\,\,   calls a function that MIGHT have side-effects}
		\State{As we mark S live, insert S.operands.definers into W}
		\While {$|W| > 0$}
		\State{S $\gets$ W.removeOne()}
		\If{{\color{blue}(S is live)}}
		\State{\textbf{continue}}
		\EndIf
		\State{{\color{blue}mark S live, insert S.operands.definers into W}}

		\EndWhile
		\EndFunction
	\end{algorithmic}
\end{algorithm}


\subsubsection{Problems within algorithm \ref{alg:Aggressive Dead Code Elimination}}
After Aggressive Dead Code Elimination applied to function  shown in \ref{fig:p58}, there is only one \texttt{return} statement left.
However, control flow is undecidable in general, so possibly the loop in \ref{fig:pp51} will iterate indefinitely and the \texttt{return } instruction will never be executed.
The problem here is we simply mark the branch statement dead.
\begin{figure}[H]
	\centering
	\begin{subfigure}{0.5\textwidth}
		\centering
		\includegraphics[width=\textwidth]{p58.pdf}
		\caption{Original code in SSA format. }
		\label{fig:p58}
	\end{subfigure}
	\begin{subfigure}{0.5\textwidth}
		\centering
		\includegraphics[width=\textwidth]{p51.pdf}
		\caption{After CCP}
		\label{fig:pp51}
	\end{subfigure}
	\begin{subfigure}{0.5\textwidth}
		\centering
		\includegraphics[width=\textwidth]{p57.pdf}
		\caption{After ADCE}
		\label{fig:p57}
	\end{subfigure}
	\caption{An example to illustrate the algorithm \ref{alg:Aggressive Dead Code Elimination} has a problem.}
	\label{fig:p51-58}
\end{figure}


Also when we apply this algorithm to \ref{fig:p55-56}, we can find that \texttt{j2 < 10} is marked dead which is wrong. Of course,
we can simply mark all branches live in the initialize stage, but this is not the ideal solution.


Now we need to carefully consider which conditional branches need to be marked live.



\subsubsection{Control Dependence}

\begin{definition}{control dependence}
	Y is control-dependent on X if
	\begin{itemize}
		\item X branches to u and v
		\item $\exists$ a path u $\rightarrow$ exit which does not go through Y
		\item $\forall$ paths v  $\rightarrow$ exit go through Y
	\end{itemize}
	This means X casn determine whether or not Y is executed.
	\begin{figure}[H]
		\centering
		\includegraphics[width=0.5\textwidth]{p59.png}

	\end{figure}
\end{definition}


\subsubsection{Aggressive Dead Code Elimination(Fixed Version)}

So we make a little modification \ref{alg:Aggressive Dead Code Elimination(Fixed Version)}. When we mark S is live, we should also mark live those conditional branches upon which S is
control dependent.

\begin{algorithm}[H]
	\caption{Aggressive Dead Code Elimination(Fixed Version)}\label{alg:Aggressive Dead Code Elimination(Fixed Version)}
	\begin{algorithmic}
		\Function{init}{}
		\State{mark as live all stmts that have side-effects:}
		\State{ \,\,\,\,\,\,\,\,    I/O}
		\State{  \,\,\,\,\,\,\,\,   stores into memory}
		\State{  \,\,\,\,\,\,\,\,   returns}
		\State{  \,\,\,\,\,\,\,\,   calls a function that MIGHT have side-effects}
		\State{As we mark S live, insert S.operands.definers into W}
		\State{\color{red} S.CD$^{-1}$ into W}
		\While {$|W| > 0$}
		\State{S $\gets$ W.removeOne()}
		\If{(S is live)}
		\State{\textbf{continue}}
		\EndIf
		\State{mark S live, insert S.operands.definers into W}
		\State{\color{red} S.CD$^{-1}$ into W}
		\EndWhile
		\EndFunction
	\end{algorithmic}
\end{algorithm}



\subsubsection{Finding the Control Dependence Graph}

\begin{itemize}
	\item Construct CFG
	\item Add entry node and exit node
	\item Add (entry, exit) edge
	\item Create G$^\prime$, the reverse CFG
	\item Compute D-tree in G$^\prime$ (post-dominators of G)
	\item Compute DF$_G^\prime$(y) for all y $\in$ G$^\prime$ (post-DF of G)
	\item Add (x,y)$\in$ G to CDG if x $\in$ DF$_G^\prime$(y)
\end{itemize}



So let us calculate the control dependence for Figure \ref{fig:pp55} which is shwon in Figure \ref{fig:p60}. Since Block1 is control dependent on Block1, so the conditional
branch in Block1 \texttt{j2 < 10 ?} should be marked live now.

\begin{figure}[H]
	\centering
	\includegraphics[width=0.6\textwidth]{p60.png}
	\caption{From left to right are G$\prime$, post-dominators of G and post-DF of G respectively.}
	\label{fig:p60}
\end{figure}






\printbibliography

\end{document}
